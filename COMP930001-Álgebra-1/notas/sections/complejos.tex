\documentclass[../main.tex]{subfiles}

\begin{document}
\section{Números Complejos}
\subsection{Raíces $n$-ésimas de la unidad} {
    Imaginemos que queremos resolver la siguiente ecuación, para $\omega \in \C$ $\land$ $n \in \Z$
    
    \begin{equation}
        \omega^n = 1
    \end{equation}
    
    Luego, el conjunto de soluciones de ésta ecuación son las \ul{raíces $n$-ésimas de la unidad.}
    
    Si reescribimos al 1 en su forma exponencial usando la \hyperref[moivre]{fórmula de De Moivre}, vemos que:
    \begin{equation}
        \omega_k = e^{\frac{2k\pi}{n}i}, \;\;\; 0 \leq k \leq n-1
    \end{equation}
    
}
\end{document}