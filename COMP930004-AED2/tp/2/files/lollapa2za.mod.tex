\documentclass[../main.tex]{subfiles}

\begin{document}
\interfaz%

\representacion%

\begin{Estructura}{lolla}[estr]
    \begin{Tupla}[estr]
        \estrMiembro{puestos}{\dicc[ID, puesto]},\\%
        \estrMiembro{personas}{\conj[persona]},\\%
        \estrMiembro{itemsTotal}{\conj[ItemsId]},\\%
        \estrMiembro{mayoresGastos}{\ColaPrioritaria[persona]},\\%
        \estrMiembro{comprasPorPersona}{\diccLog[persona, {\tupla[\estrMiembro{gastoTotal}{\nat},\\%
        \espacio{39}\estrMiembro{hackeables}{\ColaPrioritaria[\mono{compra}]},\\ \espacio{39}\estrMiembro{noHackeables}{\ColaPrioritaria[\mono{compra}]}]}]},\\%
        \estrMiembro{items}{\diccLog[itemID, {\diccLog[ID, {\Iterador[puesto]}]}]}%
    \end{Tupla}
\end{Estructura}

\begin{Tupla}
    \estrMiembro{elPuesto}{\Iterador[puesto]},
    \estrMiembro{cant}{\nat}
\end{Tupla}

\algoritmos%

\begin{implementacion}{CrearLolla}{\In{puestos}{\dicc[\puestoid, \puesto]}, \In{personas}{\conj[\persona]}}{res}{\estrLolla}
\State 
\State $res$.puestos $\gets {puestos}$  
\State $res$.personas $\gets {personas}$
\State $res$.mayoresGastos $\gets (...)$
\State $res$.items $\gets (...)$
\State $it$ $\gets {CrearIt}(personas)$
\While{$\HaySiguiente[it]$}
    \State $\text{Definir}(res.comprasPorPersona, it, \langle 0, \Vacio[\phantom{}], \Vacio[\phantom{}] \rangle)$
    \State $it \gets \Avanzar[it]$
\EndWhile
\State $it2$ $\gets {CrearIt}(claves(puestos))$
\While{$\HaySiguiente[it2]$}
    \State $it3$ $\gets {CrearIt}(it2.ObtenerMenu)$ %RECORDAR HACER LA FUNCION EN PUESTOSSSSSSSSS
    \While{$\HaySiguiente[it3]$}
        \State $\text{Agregar}(res.items, it3)$   
        \State $it3 \gets \Avanzar[it3]$
    \EndWhile
    \State $it2 \gets \Avanzar[it]$
\EndWhile



\complejidad{$\Theta(A) + \Theta(I)*\Theta(i)$} \\
\justificacion{Tenemos que construir el diccionario \textit{comprasPorPersona} que itera sobre todas las personas del festival y el diccionario \textit{items} que itera sobre todos los items del festival. Para conocer todos los items, accedo a el menu de todos los puestos, y recorro dicho menu. La cantidad de items del menu de cada puesto es \textit{i}. 
Todo lo demás viene pasado como parámetro.}

\end{implementacion}
\\


\par\vspace{4m}

\end{document}