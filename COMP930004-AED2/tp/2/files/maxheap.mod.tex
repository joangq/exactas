\documentclass[../main.tex]{subfiles}

\begin{document}
Este MaxHeap es básicamente idéntico al minHeap con dos diferencias básicas:
\begin{itemize}
    \item La propiedad de heap se cumple si el valor de todos los padres es mayor o igual al de sus hijos.
    \item Se guardan tuplas en vez naturales, y se puede modificar el valor de los nodos en $O(log\, n)$, siendo n el tamaño máximo del heap (correspondiente a la cantidad total de personas para nuestro caso de uso).
\end{itemize}

\begin{interfaz}

\textbf{se explica con: }\textsc{Árbol Binario(nodo)} \\
\textbf{géneros: \texttt{maxHeap}} \\
%\textbf{signatura: }

\InterfazFuncion{Vacio}{\In{n}{\nat}, \In{maxId}{\nat}}{\maxHeap}
[true]
{$res \igobs $nil}
[$O(n)$][Genera un heap vacío de capacidad n.]   

\InterfazFuncion{Agregar}{\Inout{h}{\maxHeap}, \In{elem}{nodo}}{}
[$h \igobs h_0 \land \text{noEstaEnElHeap}(h, \pi_1(elem)) \land \pi_1(elem) < \text{tam}(h.indicesPersonas)$]
{respetaHeap(h) $\land$ mismosElementosMasUno($h$, $h_0$, elem)}
[$O(log(n))$][Agrega un elemento al heap.] 

\InterfazFuncion{Maximo}{\In{h}{\maxHeap}}{\id}
[$\neg$nil?(h)]
{$res = \pi_2$(raiz(h))}
[$O(1)$][Devuelve la persona con mayor gasto (por como lo implementamos, siempre está en la raíz del heap).]

\InterfazFuncion{RemoverMáximo}{\Inout{h}{\maxHeap}}{}
[$h \igobs h_0 \land \neg$nil?(h)]
{respetaHeap(h) $\land$ mismosElementosMenosUno(h, $h_0$, elem)}
[$O(log(n))$][Remueve el gasto máximo del heap (por como lo implementamos, siempre está en la raíz del heap).]

{\large\bfseries Otras operaciones: }

\InterfazFuncion{ModificarGasto}{\Inout{h}{\maxHeap}, \In{persona}{\persona}, \In{nuevoGasto}{\nat}}{}
[$h \igobs h_0 \land$ persona está en el heap.]
{$h$ respeta la propiedad del heap y tiene las mismas personas y gastos que $h_0$, a excepción de la persona
pasada por parámetro, cuyo gasto será nuevoGasto.}
[$O(log(n))$][Modifica el gasto de la persona pasada como parámetro.]

\InterfazFuncion{HacerMaxHeap}{\Inout{h}{\estrHeap}, \In{i}{\nat}}{}
[$h = h_0$ y los subtrees del elemento en la posición i ya son heaps.]
{$h$ tiene los mismos elementos que $h_0$ y cumple la propiedad del maxHeap.}
[$O(log(n))$][Convierte a el subtree con raíz en la posición i de h.nodos en una representación válida de un maxHeap.]

\InterfazFuncion{Swap}{\Inout{h}{\ad[nodo]}, \In{i}{\nat}, \In{j}{\nat}}{}
[$h = h_0 \land i < \text{tamaño}(h) \land j < \text{tamaño}(h)$]
{$h$ es igual a $h_0$ con los elementos en las posiciones i y j intercambiados, modificando apropiadamente sus índices en el arreglo de índices.}
[$O(1)$][Intercambia los elementos de las posiciones i y j, modificando apropiadamente sus índices en el arreglo de índices.]

\InterfazFuncion{Izq}{\In{n}{\nat}}{\nat}
[true]
{$res =$ Izq(n)}
[$O(1)$][Devuelve el índice del hijo izquierdo del nodo en la posición n.]

\InterfazFuncion{Der}{\In{n}{\nat}}{\nat}
[true]
{$res =$ Der(n)}
[$O(1)$][Devuelve el índice del hijo derecho del nodo en la posición n.]

\InterfazFuncion{Padre}{\In{n}{\nat}}{\nat}
[true]
{$res =$ Padre(n)}
[$O(1)$][Devuelve el índice del padre del nodo en la posición n.]

{\large\bfseries Predicados auxiliares: }

respetaHeap(h : ab(nodo)) \\
mismosElementosMasUno($h_1$: ab(nodo), $h_2$: ab(nodo), a: \nat) \\
mismosElementosMenosUno($h_1$: ab(nodo), $h_2$: ab(nodo), a: \nat) \\

respetaHeap(h) $\equiv$ \\
\textbf{if} nil?(h) $\lor$ esHoja(h) \textbf{then} true \textbf{else} \\
\espacio{4} \textbf{if} $\neg$nil?(der) \textbf{then} \\
\espacio{8} $\pi_1$(raiz(h)) $\geq$ $\pi_1$(raiz(der(h))) $\land$ respetaHeap(der(h)) $\land$\\
\espacio{8} \textbf{if $\pi_1$(raiz(h)) $=$ $\pi_1$(raiz(der(h)))} \textbf{then} \\
\espacio{12} $\pi_2$(raiz(h)) $<$ $\pi_2$(raiz(der(h)) \\
\espacio{8} \textbf{endif} \\
\espacio{4} \textbf{endif} \\
\espacio{4} \textbf{if} $\neg$nil?(izq) \\
\espacio{8} $\pi_1$(raiz(h)) $\geq$ $\pi_1$(raiz(derizq)) $\land$ respetaHeap(izq(h)) $\land$\\
\espacio{8} \textbf{if $\pi_1$(raiz(h)) $=$ $\pi_1$(raiz(izq(h)))} \textbf{then} \\
\espacio{12} $\pi_2$(raiz(h)) $<$ $\pi_2$(raiz(izq(h)) \\
\espacio{8} \textbf{endif} \\
\espacio{4} \textbf{endif} \\
\textbf{endif} \\
\vspace{3mm}
mismosElementosMasUno($h_1, h_2, a) \equiv$ \\
tamaño$(h_1) =$ tamaño$(h_2) + 1$ $\land$ \\
$(\forall x: nodo) (x \neq a \implies \#\text{apariciones}(\text{inorder}(h_1), x) = \#\text{apariciones}(\text{inorder}(h_2), x))$ $\land$ \\
$\#\text{apariciones}(\text{inorder}(h_1), a) = \#\text{apariciones}(\text{inorder}(h_2), a) + 1$

mismosElementosMenosUno($h_1, h_2, a) \equiv$ \\
tamaño$(h_1) =$ tamaño$(h_2) - 1$ $\land$ \\
$(\forall x: nodo) (x \neq a \implies \#\text{apariciones}(\text{inorder}(h_1), x) = \#\text{apariciones}(\text{inorder}(h_2), x))$ $\land$ \\
$\#\text{apariciones}(\text{inorder}(h_1), a) = \#\text{apariciones}(\text{inorder}(h_2), a) - 1$
\\

noEstaEnElHeap($h, persona) \equiv$ \\
$(\forall i : \nat) (0 \leq i < h.tama\tilde{n}oActual \impluego h.nodos[i].id \neq persona)$

{\large\bfseries Funciones auxiliares: } \\
($\forall$ sec: secu($nodo$)) \\
($\forall$ a: $nodo$)

\#apariciones : secu($nodo$) sec $\times nodo$ a $\longleftarrow$ nat\\
Izq, Der, Padre (definidas en la parte de representación).

$\#$apariciones($sec, a$) $\equiv$ \\
\textbf{if} vacia?(sec) \textbf{then} 0 \textbf{else} \\
\espacio{4} \textbf{if} prim$(sec) = a$ \textbf{then} 1 + $\#$apariciones$($fin$(sec, a))$ \\
\espacio{4} \textbf{else} 0 + $\#$apariciones$($fin$(sec, a))$ \\
\textbf{endif}

\end{interfaz}

\begin{representacion}
Se considera que un nodo esta vacío cuando su id es 0, y para que no haya conflictos con una persona con id 0, al guardar el id se le suma 1, y al devolverlo se le resta 1.

Además, guardamos los índices de las personas (aquellas que están en el arreglo dimensionable de nodos) ya que nos permite tener una complejidad temporal de peor caso logarítmica a la hora de modificar un nodo (ya que, sino, tendríamos que buscar la posición del nodo a modificar, lo que nos llevaría $O(n)$).

\begin{Estructura}{heap}[estrHeap]
    \begin{Tupla}
        \estrMiembro{nodos}{\ad[nodo]},\\%
        \estrMiembro{indicesPersonas}{\ad[\nat]},\\%
        \estrMiembro{tamañoActual}{\nat}
    \end{Tupla}

    \begin{Tupla}[nodo]
        \estrMiembro{gasto}{\nat},\\%
        \estrMiembro{id}{\persona}
    \end{Tupla}
\end{Estructura}

Padre : \nat $\longrightarrow$ \nat\\
Padre($i$) $\equiv$ Div(i - 1, 2)

Izq : \nat $\longrightarrow$ \nat\\
Izq($i$) $\equiv 2\; \times $ i $\; + 1$

Der : \nat $\longrightarrow$ \nat\\
Der($i$) $\equiv 2\; \times $ i $\; + 2$

Div : \nat \;n $\; \times$ \nat \;k $\longrightarrow$ \nat\\
Div(n, k) $\equiv$ \textbf{if} n $<$ k \textbf{then} 0 \textbf{else} 1 $+$ div(n $-$ k, k) \textbf{fi}

Rep : estrHeap $\longrightarrow$ bool\\
Rep(h) $\equiv$ true $\iff$ esHeap(h.nodos, 0) $\land$ coincidenIndices(h.nodos, h.indicesPersonas, h.tamañoActual)
% nodosNoNulos(h.nodos, 0) $=$ h.tamañoActual $\yluego$ 

esHeap : \ad[nodo] h $\times$ \nat[] i $\longrightarrow$ \bool \hfill \{i $<$ tam(h)\} \\
esHeap($h, i$) $\equiv$\\
$(h.nodos[i].id = 0 \lor (h.nodos[\text{Izq}(i)].id = 0 \land h.nodos[\text{Der}(i)].id = 0)) \;\lor$ ( \vspace{2mm} \hfill \\
\espacio{4} $(h.nodos[\text{Der}(i)].id \neq 0 \impluego h.nodos[i].gasto \geq h.nodos[\text{Der}(i)].gasto \land esHeap(h.nodos, \text{Der}(i)) \;\land$ \\
\espacio{4} $(h.nodos[\text{Izq}(i)].id \neq 0 \impluego h.nodos[i].gasto \geq h.nodos[\text{Izq}(i)].gasto \land esHeap(h.nodos, \text{Izq}(i)) \;\land$ \\
\espacio{4} $(h.nodos[\text{Izq(i)}.id \neq 0 \land h.nodos[i].gasto = h.nodos[\text{Izq}(i)].gasto \implies h.nodos[i].id < h.nodos[\text{Izq}(i)].id) \;\land$ \\
\espacio{4} $(h.nodos[\text{Der(i)}.id \neq 0 \land h.nodos[i].gasto = h.nodos[\text{Der}(i)].gasto \implies h.nodos[i].id < h.nodos[\text{Der}(i)].id)$ \\
$)$

coincidenIndices : \ad[nodo] $\times$ \diccLog[\persona, \nat] $\times$ \nat \\
coincidenIndices(n, ind, tamActual) $\equiv$\\
$(\forall i: \nat) (0 \leq i < tamActual \impluego ind[n[i].id - 1] = i)$

% nodosNoNulos : \ad[\nodo] $\longrightarrow$ $\nat$\\
% nodosNoNulos(n, i) $\equiv$ \\
% \textbf{if} i $\geq$ tam(n) \textbf{then} 0 \textbf{else}\\
% \espacio{4} \textbf{if} n[i] = $\langle 0, 0 \rangle$ \textbf{then} 1 + nodosNoNulos(n, i + 1) \\
% \espacio{4} \textbf{else} 0 + nodosNoNulos(n, i + 1) \\
% \espacio{4} \textbf{fi} \\
% \textbf{fi}

Abs : estrHeap h $\longrightarrow$ \ab[nodo] \hfill \{Rep(h)\}\\
Abs(h) $\equiv$ CrearArbol(h.nodos, 0)

CrearArbol : \ad[nodo] h $\times$ \nat[] i $\longrightarrow$ \ab[nodo] \hfill \{i $<$ tam(h)\} \\
CrearArbol(h, i) $\equiv$ \textbf{if} h[i] $= 0$ \textbf{then} nil \textbf{else} bin(CrearArbol(h, Izq(i)), h[i], CrearArbol(h, Der(i)))
\end{representacion}

\begin{algoritmos}
\begin{implementacion}{Vacio}{\In{n}{\nat}, \In{maxid}{\nat}}{res}{\estrHeap}
\State $arr \gets crearArreglo(n)$ \Complexity{$O(1)$}
\State $i \gets 0$ \Complexity{$O(1)$}
\While{$i < n$} \Complexity{$O(1 * n)$}
    \State $arr[i] \gets \langle 0, 0 \rangle$ \Complexity{$O(1)$}
    \State $i \gets i + 1$ \Complexity{$O(1)$}
\EndWhile
\State $indicesP \gets crearArreglo(maxid + 1)$ \Complexity{$O(1)$}
\State $res \gets \langle arr, indicesP, 0 \rangle$ \Complexity{$O(1)$}

\complejidad{$O(n)$}\\
\justificacion{Creamos n posiciones. Inicializar cada posición cuesta $O(1)$. Todo lo demás es $O(1)$.}
\end{implementacion}

\begin{implementacion}{Agregar}{\Inout{h}{\maxHeap}, \In{n}{\nodo}}{}{}
\State $i \gets h.tama\tilde{n}oActual$ \Complexity{$O(1)$}
\State $copia\_n \gets n$ \Complexity{$O(1)$}
\State $copia\_n.id \gets copia\_n.id + 1$ \comentario{explicado en aclaraciones} \Complexity{$O(1)$}
\State $h.nodos[i] \gets copia\_n$ \Complexity{$O(1)$}
\State $h.indicesPersona[n.id] \gets i $ \Complexity{$O(1)$}
\State $h.tama\tilde{n}oActual \gets h.tama\tilde{n}oActual + 1$ \Complexity{$O(1)$}
\\
\While{$(i \neq 0) \land (h.nodos[i].gasto > h.nodos[\text{Padre}(i)].gasto)$} \Complexity{$O(log(n))$}
    \State $j \gets \text{Padre}(i)$ \Complexity{$O(1)$}
    \State $\text{Swap}(h, i, j)$ \Complexity{$O(1)$}
    \State $i \gets j$ \Complexity{$O(1)$}
\EndWhile
\\
\State $j \gets \text{Padre}(i)$ \Complexity{$O(1)$}
\\
\While{$(i \neq 0) \land (h.nodos[i].gasto = h.nodos[j].gasto) \land (h.nodos[i].id > h.nodos[j].id)$} \Complexity{$O(log(n))$}
    \State $j \gets \text{Padre}(i)$ \Complexity{$O(1)$}
    \State $\text{Swap}(h, i, j)$ \Complexity{$O(1)$}
    \State $i \gets j$ \Complexity{$O(1)$}
\EndWhile

\complejidad{$O(log(n))$}\\
\justificacion{Agrego el nodo al final del heap y hago sift up, lo que lleva $O(log(n)).$ Cuando corta el sift up ``convencional'', sigo haciendo sift up mientras los gastos sean iguales pero el id del nodo ingresado sea menor que el de su padre (también lleva $O(log(n))$). En conclusión, tenemos dos ciclos que a lo sumo nos llevan $O(log (n))$, y todas las demás operaciones son $O(1)$, así que la complejidad final es $O(log(n))$.}
\end{implementacion}

\begin{implementacion}{Maximo}{\In{h}{\maxHeap}}{res}{\persona}
\State $res \gets h.nodos[0].id + 1$ \Complexity{$O(1)$}

\complejidad{$O(1)$}\\
\justificacion{Únicamente accedo al valor del nodo y le sumo 1. La suma es una operacion elemental y cuesta $O(1)$}
\end{implementacion}

\begin{implementacion}{RemoverMaximo}{\Inout{h}{\maxHeap}}{}{}
\State $h.nodos[h.tama\tilde{n}oActual] \gets \langle 0, 0\rangle$ \Complexity{$O(1)$}
\State $h.tama\tilde{n}oActual \gets h.tama\tilde{n}oActual - 1$ \Complexity{$O(1)$}
\State $h.nodos[0] \gets h.nodos[h.tama\tilde{n}oActual]$ \Complexity{$O(1)$}
\State $h.nodos.\text{HacerMaxHeap}(0)$ \Complexity{$O(log(n))$} 

\complejidad{$O(log(n))$}\\

\justificacion{Pongo el último elemento en la raíz y hago sift down, lo que lleva $O(log(n)).$}
\end{implementacion}

\begin{implementacion}{ModificarGasto}{\Inout{h}{\maxHeap}, \In{persona}{\persona}, \In{nuevoGasto}{\nat}}{}{}
    \State $i \gets h.indicesPersona[persona]$ \Complexity{$O(1)$}
    \\
    \If {$h.nodos[i].gasto < nuevoGasto$} \Complexity{$O(1)$}
        \State $h.nodos[i] \gets nuevoGasto$ \Complexity{$O(1)$}
        \State $h.HacerMaxHeap(i)$ \Complexity{$O(log(n))$}
    \ElsIf{$h.nodos[i].gasto > nuevoGasto$} \Complexity{$O(1)$}
        \While{$i \neq 0 \land h.nodos[i].gasto <  h.nodos[\text{Padre}(i)].gasto$} \Complexity{$O(1 * log(n))$}
            \State $j \gets \text{Padre}(i)$ \Complexity{$O(1)$}
            \State $\text{Swap}(h, i, j)$ \Complexity{$O(1)$}
            \State $i \gets j$ \Complexity{$O(1)$} 
        \EndWhile
    \EndIf

\complejidad{$O(log(n))$}\\
\justificacion{Tomo el índice de la persona en el arreglo, y en base a si el valor nuevo
es mayor o menor que el actual, realizo sift down o sift up (respectivamente).}
\end{implementacion}

\begin{implementacion}{HacerMaxHeap}{\Inout{h}{\estrHeap}, \In{i}{\nat}}{}{}
\State $izq \gets$ Izq(i) \Complexity{$O(1)$}
\State $der \gets$ Der(i) \Complexity{$O(1)$}
\State $mayor \gets i$ \Complexity{$O(1)$}
\\
\If{$(izq < h.tama\tilde{n}oActual) \land (h.nodos[izq].gasto > h.nodos[mayor].gasto)$} \Complexity{$O(1)$}
    \State $mayor \gets izq$ \Complexity{$O(1)$}
\EndIf
\\
\If{$(der < h.tama\tilde{n}oActual) \land (h.nodos[der].gasto > h.nodos[mayor].gasto)$} \Complexity{$O(1)$}
    \State $mayor \gets der$ \Complexity{$O(1)$}
\EndIf
\\
\If{$mayor \neq i$} \Complexity{$O(1)$}
    \State $\text{Swap}(h, i, mayor)$ \Complexity{$O(1)$}
    \State $\text{HacerMaxHeap}(h, mayor)$ \Complexity{$O(log(n))$}
\EndIf

\complejidad{$O(log(n))$}\\
\justificacion{Sift down lleva $O(log(n)).$}
\end{implementacion}

\begin{implementacion}{Swap}{\Inout{h}{\estrHeap}, \In{i}{\nat}, \In{j}{\nat}}{}{}
\State $tempIndice \gets h.nodos[i].id$ \Complexity{$O(1)$}
\State $h.indicesPersonas[h.nodos[i].id - 1] \gets h.indicesPersonas[h.nodos[j].id - 1]$ \Complexity{$O(1)$}
\State $h.indicesPersonas[h.nodos[j].id - 1] \gets tempIndice - 1$ \Complexity{$O(1)$}
\\
\State $temp \gets h.nodos[i]$ \Complexity{$O(1)$}
\State $h.nodos[i] \gets h.nodos[j]$ \Complexity{$O(1)$}
\State $h.nodos[j] \gets temp$ \Complexity{$O(1)$}


\complejidad{$O(1)$}\\
\justificacion{Asignar y modificar solo cuesta $O(1)$.}
\end{implementacion}

\begin{implementacion}{Izq}{\In{i}{\nat}}{\nat}{}
\State $res \gets 2 \times i + 1$ \Complexity{$O(1)$}

\complejidad{$O(1)$}\\
\justificacion{Solo se utilizan operaciones elementales, las cuales cuestan $O(1)$}
\end{implementacion}

\begin{implementacion}{Der}{\In{i}{\nat}}{\nat}{}
\State $res \gets 2 \times i + 2$ \Complexity{$O(1)$}
\complejidad{$O(1)$}\\
\justificacion{Solo se utilizan operaciones elementales, las cuales cuestan $O(1)$}
\end{implementacion}

\begin{implementacion}{Padre}{\In{i}{\nat}}{\nat}{}
\State $res \gets (i - 1) / 2$ \Complexity{$O(1)$}

\complejidad{$O(1)$}\\
\justificacion{Solo se utilizan operaciones elementales, las cuales cuestan $O(1)$}
\end{implementacion}
\end{algoritmos}
\end{document}