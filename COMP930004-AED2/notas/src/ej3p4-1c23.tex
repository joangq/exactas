\documentclass{article}
\usepackage{xcolor}
\usepackage{pdflscape}
\usepackage{everypage}
\usepackage{lipsum}
\usepackage{fancyhdr}
\usepackage{amsmath}
\usepackage{amssymb}
\usepackage[colorlinks]{hyperref}
\usepackage[paperheight=297mm, paperwidth=210mm,
includehead,
nomarginpar,
textwidth=15cm,
headheight=0mm,
]{geometry}

\setlength{\parindent}{0cm}

\newcommand{\doble}[2]{\def\arraystretch{1.25}\begin{tabular}{@{}c@{}}#1\\#2\end{tabular}}

\begin{document}\pagenumbering{gobble}

\newcommand{\daga}{$^\dagger$}
\newcommand{\ddaga}{$^\ddagger$}
\newcommand{\uno}{\hyperref[uno]{\daga}}
\newcommand{\dos}{\hyperref[dos]{\ddaga}}

\hypersetup{
    colorlinks,
    citecolor=black,
    filecolor=black,
    linkcolor=black,
    urlcolor=black
}

\pagestyle{fancy}
\fancyhead{}
\fancyhead[R]{\textbf{Cota superior de complejidad temporal en peor caso por estructura}}
\fancyfoot{}
\fancyfoot[R]{%
\begin{minipage}[t]{0.48\textwidth}
\begin{flushright}
\href{https://github.com/joangq/exactas}{GitHub $\triangleleft$} \\
\href{https://github.com/joangq/exactas/blob/main/COMP930004-AED2/notas/src/ej3p4-1c23.tex}{\LaTeX{ }Source $\triangleleft$} \\
\end{flushright}
\end{minipage}
}

\begin{table}[!ht]
\centering
\bgroup
\def\arraystretch{2.0}
  \begin{tabular}{|c|c|c|c|c|c|c|} 
	  \hline
	  Operación & \doble{Lista}{Enlazada} & \doble{Lista Enlazada}{Ordenada} & \doble{ABB}{(BST)} & AVL & Heap & Trie \\

	  \hline\hline

	  \multicolumn{1}{|c|}{Pertenece}			& 
	  \multicolumn{1}{ c }{$O(n)$}				& 
	  \multicolumn{1}{ c }{$O(n)$}				& 
	  \multicolumn{1}{ c }{$O(n)$}				& 
	  \multicolumn{1}{ c }{$O(\log n)$}			& 
	  \multicolumn{1}{ c }{$O(n)$}				& 
	  \multicolumn{1}{ c|}{$O(|k|)$}			\\
	  
	  \multicolumn{1}{|c|}{Inserción}			& 
	  \multicolumn{1}{ c }{$O(1)$\uno}			& 
	  \multicolumn{1}{ c }{$O(1)$\uno}			& 
	  \multicolumn{1}{ c }{$O(n)$}				& 
	  \multicolumn{1}{ c }{$O(\log n)$}			& 
	  \multicolumn{1}{ c }{$O(\log n)$}			& 
	  \multicolumn{1}{ c|}{$O(|k|)$}			\\

	  \multicolumn{1}{|c|}{Borrado}				& 
	  \multicolumn{1}{ c }{$O(1)$\uno}			& 
	  \multicolumn{1}{ c }{$O(1)$\uno}			& 
	  \multicolumn{1}{ c }{$O(n)$}				& 
	  \multicolumn{1}{ c }{$O(\log n)$}			& 
	  \multicolumn{1}{ c }{$O(n)$}				& 
	  \multicolumn{1}{ c|}{$O(|k|)$}			\\

	  \multicolumn{1}{|c|}{Buscar Mínimo}		& 
	  \multicolumn{1}{ c }{$O(n)$}				& 
	  \multicolumn{1}{ c }{$O(1)$\dos}			& 
	  \multicolumn{1}{ c }{$O(n)$}				& 
	  \multicolumn{1}{ c }{$O(\log n)$}			& 
	  \multicolumn{1}{ c }{$O(1)$\dos}			& 
	  \multicolumn{1}{ c|}{$O(n)$}				\\

	  \multicolumn{1}{|c|}{Borrar Mínimo}		& 
	  \multicolumn{1}{ c }{$O(n)$}				& 
	  \multicolumn{1}{ c }{$O(1)$\uno\dos}		& 
	  \multicolumn{1}{ c }{$O(n)$}				& 
	  \multicolumn{1}{ c }{$O(\log n)$}			& 
	  \multicolumn{1}{ c }{$O(\log n)$\dos}		& 
	  \multicolumn{1}{ c|}{$O(|k|)$}			\\

	  %\multicolumn{1}{|c|}{Buscar Máximo}		& 
	  %\multicolumn{1}{ c }{$O()$}				& 
	  %\multicolumn{1}{ c }{$O()$}				& 
	  %\multicolumn{1}{ c }{$O()$}				& 
	  %\multicolumn{1}{ c }{$O()$}				& 
	  %\multicolumn{1}{ c }{$O()$}				& 
	  %\multicolumn{1}{ c|}{$O()$}				\\
	  
	  \hline
  \end{tabular}
\egroup
\end{table}

Siendo $n$ la cantidad de elementos presentes en la colección. Y, en el caso del trie, siendo $|k|$ el largo de la clave (\textit{key}).

\vfill

{\label{uno} \setlength{\parindent}{-1mm} 
\daga Considerando que, dado un iterador (puntero) al elemento, la inserción/remoción como tal --es decir, reacomodar los punteros de la estructura-- se realiza en $O(1)$. Buscar el elemento es $O(n)$.}

\par
\vspace{5mm}

{\label{dos} \setlength{\parindent}{-1mm} 
\ddaga Considerando que la estructura preserva el orden total de \textit{menor o igual.} ($\bullet \leqslant \bullet $). Es decir, dados $a$, $b$ elementos pertenecientes a la estructura, $a$ aparece antes que $b$ si y sólo si $a < b$. Si $a = b$ entonces $b$ podría aparecer antes que $a$. 
\\ \\
En caso de la lista enlazada, significa que está ordenada de forma ascendente. En cambio, si se trata de un heap, entonces es un $min$-heap.
}

\hypersetup{
    colorlinks,
    citecolor=blue,
    filecolor=blue,
    linkcolor=blue,
    urlcolor=blue
}

\end{document}
