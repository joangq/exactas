\documentclass[../main.tex]{subfiles}

\begin{document}
\part[Fórmulas y Definiciones]{\decorate{Fórmulas y Definiciones}}

\section{Grupos y Anillos} {
    \subsection{Definición -- Grupo} {
        
    }
    
    \subsection{Definición -- Orden} {
        Sea $G$ un grupo. Sea $a$ un elemento de $G$, tal que $a \in G$, definimos:
        \begin{enumerate}
            \item {
                $|G|$ es la cantidad de elementos en $G$.
                \\ Decimos que es el \textit{orden} o la \textit{cardinalidad} de $G$
                \nln \ul{Ejemplos:} 
                \begin{itemize}
                    \item $|\Z| = \infty$
                    \item $|\Z/9\Z| = $\#\{$\bar{0},\bar{1},\bar{2},\bar{3},\bar{4},\bar{5},\bar{6},\bar{7},\bar{8}$\}$ = 9$
                \end{itemize}
            }
            
            \item $|a|$ es el entero positivo más chico tal que $a^n$ sea el elemento identidad. %es decir $na = 0$%
            Decimos que es el \textit{orden} o el \textit{periodo} de $a$.
            
            \nln
            
            \ul{Ejemplo:}
            
            \begin{itemize}
                \item En $\Z/12\Z$, $|5|$ = 2, pues $5^{2} \equiv$ 1 (mod 12)
            \end{itemize}
            
            
        \end{enumerate}
    }
}
    
    \subsection{Definición -- Anillo} {
        Sea $R$ un anillo. Luego $R$ es un conjunto, donde se definen dos operaciones binarias: \textit{adición} y \textit{multiplicación}, tales que \ul{para todo $a$, $b$, $c$ en $R$} se cumpla:
            \begin{enumerate}
                \item { \textbf{Conmutatividad de la adición:} \\ 
                $a + b = b + a$ }
                
                \item { \textbf{Asociatividad de la adición} \\ 
                $(a+b)+c = a + (b+c)$ }
                
                \item { \textbf{Existencia de neutro aditivo} \\
                Existe un elemento 0 en $R$ tal que $a + 0 = a$  $\forall a \in R$ }
                
                \item { \textbf{Existencia de inverso aditivo} \\
                $\exists$ $-a$ tal que $ a + (-a) = 0$ }
                
                \item { \textbf{Asociatividad de la multiplicación} \\ $a (bc) = (ab) c$ }
                
                \item { \textbf{Distribución de la multiplicación respecto de la adición} \\ $a(b+c) = ab + ac \land (b+c)a = ba+ca$ }
            \end{enumerate}
            
        \subsubsection{Conmutatividad de la multiplicación} {
            Sea $R$ un anillo. Luego, si \ul{para todo $a$, $b$, $c$ en $R$} se cumple: $ab = ba$ \\ Entonces se dice que $R$ es un \textit{anillo conmutativo}.
        }
        
        \subsubsection{Elemento identidad} {
            Sea $R$ un anillo, y sea $e$ un elemento de $R$, luego si $e \neq 0$ y
            se cumple que
            \begin{equation*} { ae = a \;\; \forall a \in R } \end{equation*}
            Entonces decimos que $e$ es el \textit{elemento identidad}. En particular, es la identidad de la multiplicación. También recibe el nombre de \textit{unidad} ó \textit{unity}, en inglés.
        }
        
        \subsubsection{Elemento inversible} {
            Sea $R$ un anillo, y sea $a$ un elemento de $R$, $a \neq 0$. Luego si existe un elemento $a^{-1} \in R$ tal que
            \begin{equation*} { a \cdot a^{-1} = e} \end{equation*}
            Donde $e$ es el elemento identidad de $R$. \nln
            Luego, $a^{-1}$ es el \textit{inverso multiplicativo} de $a$. Decimos que $a$ es \textit{inversible}. 
            
            
            \nln Además, se cumple que:
            
            \begin{enumerate}
                \item El inverso multiplicativo es único para cada inversible de $R$.
                \item El conjunto de los inversibles en $R$ forman un grupo. Se suele notar $U(R)$.
            \end{enumerate}
            
            \nln
            \ul{Nota:} Otro nombre para los elementos inversibles es \textit{unidad} ó \textit{unit}, en inglés, diferenciándose de \textit{unity}, que es el nombre que recibe el elemento identidad. Es por eso que \ul{en éste texto \textit{unidad} sólo refiere al elemento identidad.}
            
        }
        
    } 

\newpage

\section{Números Naturales} {
    \subsection{Suma de Gauss} {
        Para todo $n \in \N$ vale que:
        \begin{equation*}
           \sum_{k=1}^n {k} =  1 + 2 + \cdots + (n-1) + n = \frac{n(n+1)}{2}
        \end{equation*}
    } \label{suma_gauss}
    
    \subsection{Serie geométrica} {
        Para todo $n \in \N$ y algún $q \in \N$ vale que:
        \begin{equation*}
            \sum_{i=0}^n {q^i} = q^0 + q^1 + q^2 + \cdots + q^n =
            \left\{
                \begin{array}{lr}
                    n+1, & \text{si } q = 1 \\\\
                    \frac{q^{n+1}-1}{q-1} & \text{si } q \neq 1
                \end{array}
            \right.
        \end{equation*}
    } \label{serie_geometrica}
    
    
}
\newpage

\section{Combinatoria} {
    \subsection{Cardinalidad de un conjunto} {
        
    }
    
    \subsection{Cantidad de funciones inyectivas} {
        \cite{teresa} Sean $A_m$ y $B_n$ conjuntos finitos, con $m$ y $n$ elementos respectivamente, donde $m \leq n$ . Entonces la cantidad de funciones inyectivas $f : A_m \rightarrow B_n$
        que hay es:
            \begin{equation*}
                n \cdot (n-1) \cdots (n-m+1) = \frac{n!}{(n-m)!}
            \end{equation*}
            
        Cabe mencionar que no hay una fóormula tan simple como las anteriores para
        contar la cantidad de funciones sobreyectivas que hay de un conjunto $A_n$ de
        $n$ elementos en un conjunto $B_m$ de $m$ elementos, con $n \geq m$ cualesquiera.
        Existen fórmulas pero son más complicadas e involucran en general
        contar la cantidad de elementos de muchos conjuntos.
    } \label{cantidad_inyectivas}
    
    \subsection{Número combinatorio} {
        Sea $n \in N_0$ y sea $A_n$ un conjunto con $n$ elementos. Para $0 \leq k \leq n$ , la
        cantidad de subconjuntos con $k$ elementos del conjunto $A_n$, o  equivalentemente, la cantidad de maneras que hay de elegir $k$ elementos en el conjunto $A_n$, es:
            \begin{equation*}
                \begin{pmatrix} n\\k \end{pmatrix} = \frac{n!}{k!(n-k)!}
            \end{equation*}
            
        \subsubsection{Forma recursiva} {
            \begin{equation*}
                \begin{pmatrix} n+1\\k \end{pmatrix} = 
                \begin{pmatrix} n\\k-1 \end{pmatrix} +
                \begin{pmatrix} n\\k \end{pmatrix}
            \end{equation*}
        }
        
}

    \subsection{Binomio de Newton} {
        La fórmula del binomio de Newton produce polinomios con los coeficientes
        correspondientes a los números del triángulo de Pascal, tal que:
        \begin{equation*}
            (x+y)^n = \sum_{k=0}^n \begin{pmatrix}n\\k\end{pmatrix} x^k y^{n-k}
        \end{equation*}
        
        \nln \ul{Ejemplos:}
        \begin{align*}
            (x+y)^0 =& \;1                                                  \\
            (x+y)^1 =& \;x+y                                                \\
            (x+y)^2 =& \;x^2+2xy+y^2                                        \\
            (x+y)^3 =& \;x^3+3x^2y+3xy^2+y^3                                \\
            (x+y)^4 =& \;x^4 + 4x^3y +6x^2y^2 + 4xy^3 + y^4                 \\
            (x+y)^5 =& \;x^5 + 5x^4y + 10x^3y^2 +10x^2y^3 + 5xy^4 + y^5
        \end{align*}
    }
}
\newpage
\section{$\Z$ y $\Z / p\Z$} {

\subsection{Ecuaciones de Congruencia} {
Para $a,c \in \Z$, las ecuaciones de la forma
\begin{equation*}
    aX \equiv c \; \text{(mod m)}
\end{equation*}

Tienen solución si y sólo si $(a:m)=1$
}
 \label{lcez}
 
\subsection{Inversibles en $\Z/n\Z$} {
    \cite{teresa} es un Corolario de las \hyperref[lcez]{Ecuaciones de Congruencia.} \\
    Sea $n \in \N$ y $r \in \Z /n \Z$ entonces $n$ es inversible \ul{si y solo si} $r$ es coprimo con $n$. 
    
    \nln Es decir:
    \begin{equation*}
        \text{$r$ inversible mod $n$ $\iff$ $rx = 1$ (mod $n$) tiene solución $\iff$ $(r:n)=1$}
    \end{equation*} 
    
    \nln
    Un corolario de esto es que si $r,p \in \N$ y $p primo$, luego $r$ siempre es inversible módulo $p$.

} \label{inversibles_znz}

\subsection{Pequeño Teorema de Fermat} {
\begin{enumerate}
    \item { Dados $a$, $p$ $\in \Z$ y $p$ primo, entonces:
        \begin{equation*}
            a^p \equiv a \hspace{1mm} \text{(mod p)}
        \end{equation*}
    }
    \item { Además, si $p \notdivides a$
        \begin{equation*}
            a^{p-1} \equiv 1 \hspace{1mm} \text{(mod p)}
        \end{equation*}
    }
\end{enumerate}
}

\subsection{Función $\varphi (n)$ de Euler} {
Dado $n \in \N$, entonces:
\begin{equation*}
    \varphi (n) = n \prod_{p \divides n}{1 - \frac{1}{p}}
\end{equation*}
}

\subsection{Teorema de Euler} {
Si $a$, $n$ $\in \N$ tal que $(a : n) = 1$, entonces
\begin{equation*}
    a ^ {\varphi (n)} \equiv 1 \text{ (mod n)} 
\end{equation*}
}

\subsection{Inverso multiplicativo módulo n} {
Si $a$, $n$ $\in \N$ tal que $(a : n) = 1$ y $a$ inversible, entonces
\begin{equation*}
    a^{-1} \equiv a ^ {\varphi (n) -1} \text{ (mod n)} 
\end{equation*}

\ul{Demostración:} Por Teorema de Euler sabemos que:

\begin{equation*}
    a ^ {\varphi (n)} \equiv 1 \text{ (mod n)} 
\end{equation*}

Luego, multiplicando todo por $a^-1$ obtenemos:

\begin{equation*}
    a^{-1} \cdot a ^ {\varphi (n)} \equiv 1 \cdot {a^{-1}} \text{ (mod n)} 
\end{equation*}

\begin{equation*}
    a ^ {\varphi (n) -1} \equiv {a^{-1}} \text{ (mod n)} 
\end{equation*}

} \label{inverso_mod_n}

\subsection{Potencias módulo m} {
Dados $a,n,m \in \N$, si $a$ es invertible módulo $m$, entonces
\begin{equation*}
    a^n \equiv 0 \T{(mod m)}
\end{equation*}

Nunca tiene soluciones.
} \label{potencias_mod_n}

}

\section{Números Complejos} {
    \subsection{Fórmula de De Moivre} {
        
    } \label{moivre}
}
\end{document}