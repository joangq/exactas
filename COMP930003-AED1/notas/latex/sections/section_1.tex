\documentclass[../main.tex]{subfiles}

\begin{document}

% TODO: Mover esto a un archivo aparte!
\section{Especificación de Programas}

La especificación de programas es la forma que tenemos de describir \text{qué} problema queremos resolver, sin describir \textit{cómo} queremos
resolverlo.

\subsection{Subespecificación y Sobreespecificación}

Dada una especificación con una precondición $P$ y una postcondición $Q$ consideremos a $X = \{ x_1, x_2, x_3, ..., x_n\}$ el conjunto de todos valores de entrada de un programa, y $P_X$ el conjunto de predicados que
cumplen la precondición $P$, podemos definir lo siguiente:

\begin{itemize}
    \item El programa está \ul{sobreespecificado} si $\exists x_i \in X \mid P(x_i) = \text{True} \land Q(x_i) = \text{False}$ 
          \begin{gather*}
            P_X = X = \{
            \color{dark_green}\underset{\substack{\cmark}}{x_1}\color{black},
            \color{red}\underset{\substack{\xmark\\\T{Está}\\\T{\textit{``de más"}}}}{x_2}\color{black},
            \color{dark_green}\underset{\substack{\cmark}}{x_3}\color{black},
            \color{dark_green}\underset{\substack{\cmark}}{x_4}\color{black},
            \color{dark_green}\underset{\substack{\cmark}}{x_5}\color{black},
            \color{dark_green}\underset{\substack{\cmark}}{x_6}\color{black},
            \color{dark_green}\underset{\substack{\cmark}}{x_7}\color{black},
            \color{dark_green}\underset{\substack{\cmark}}{x_8}\color{black},
            \color{dark_green}\underset{\substack{\cmark}}{x_9}\color{black},
            \color{dark_green}\underset{\substack{\cmark}}{x_{10}}\color{black},
            \dots,
            \color{dark_green}\underset{\substack{\cmark}}{x_n}\color{black}\}
            \\
            Q(x_1) = \T{True},\;Q(x_2) = \T{False},\;Q(x_3) = \T{True}\;\cdots\;Q(x_n) = \T{True}
          \end{gather*}
    \item El programa está \ul{subespecificado} si $\exists x_i \in X \mid Q(x_i) = \T{True} \land x_i \notin P_X$
          \begin{gather*}
            X= \{
            \color{dark_green}\underset{\substack{\cmark}}{x_1}\color{black},
            \color{dark_green}\underset{\substack{\cmark}}{x_2}\color{black},
            \color{dark_green}\underset{\substack{\cmark}}{x_3}\color{black},
            \color{dark_green}\underset{\substack{\cmark}}{x_4}\color{black},
            \color{dark_green}\underset{\substack{\cmark}}{x_5}\color{black},
            \color{dark_green}\underset{\substack{\cmark}}{x_6}\color{black},
            \color{dark_green}\underset{\substack{\cmark}}{x_7}\color{black},
            \color{dark_green}\underset{\substack{\cmark}}{x_8}\color{black},
            \color{dark_green}\underset{\substack{\cmark}}{x_9}\color{black},
            \color{dark_green}\underset{\substack{\cmark}}{x_{10}}\color{black},
            \dots,
            \color{dark_green}\underset{\substack{\cmark}}{x_n}\color{black}\}
            \\
            Q(x_i) = \T{True}\;\;\forall x_i \in X
            \\
            P_X = \{x_1, x_2, x_3\} \leftrightsquigarrow \T{Es \textit{``más chico''}}
          \end{gather*}
\end{itemize}


\section{Corrección de Programas}

\subsection{SmallLang}

Para poder hablar de corrección de \ul{programas}, introducimos un lenguaje imperativo simplificado.
Usamos SmallLang \cite{gries}, que nos permite dar instrucciones y controlar el flujo de ejecución de dichas instrucciones
con estructuras de control.

\nln

\ul{Instrucciones:}
\begin{itemize}
    \item Asignación -- \mono{x := E} \\
          Le asignamos a \mono{x} el valor de la expresión \mono{E}. \\
          Por ejemplo: \mono{ejemplo := a*b + 2}
    
    \item Nada -- \mono{skip}. \\
          La instrucción indica que no debe ejecutarse nada.
\end{itemize}

De esta manera, \textbf{una instrucción es un programa}. Supongamos que \mono{S1} y \mono{S2}
son programas, entonces \mono{S1; S2} es un programa también. Además, si \mono{B} es una expresión
lógica, podemos usar estructuras de control.

\nln

\ul{Estructuras de control:}
\begin{itemize}
    \item Condicional -- \mono{if B then S1 else S2 endif} es un programa. 
    \item Ciclo -- \mono{while B do S1 endwhile} es un programa.
\end{itemize}

\nln

Un ejemplo de programa puede ser: \\

\begin{center}
    \begin{tabular}{l}
        \mono{x := 0}                      \\
        \mono{while (x $>\;$0)} \\ 
        \mono{x := x+1;}                   \\ 
        \mono{x := x*2}                    \\ 
        \mono{endwhile}
    \end{tabular}
\end{center}

\nln

A medida que el programa se ejecuta, el valor de la variable \mono{x} va cambiando. Al conjunto de todos
los predicados que pueden ser verdaderos o falsos, según las variables del programa, lo llamamos \textbf{estado}.
De esta manera, la \textbf{ejecución} de un programa es una sucesión de estados. \nln

Por ejemplo, veamos los estados del siguiente programa: \nln


\begin{center}
    \begin{tabular}{l}
        \textcolor{blue}{\mono{\textbraceleft a = \ensuremath{A_0}\textbraceright}} \\
        \mono{c := a+2} \\
        \textcolor{blue}{\mono{\textbraceleft a = \ensuremath{A_0} \y\;c = \ensuremath{A_0} + 2\textbraceright}} \\
        \mono{result := c-1} \\
        \textcolor{blue}{\mono{\textbraceleft a = \ensuremath{A_0} \y\;c = \ensuremath{A_0} + 2 \y\;result = (\ensuremath{A_0} + 2) - 1 = \ensuremath{A_0} + 1\textbraceright}} \label{correct_program_example}
    \end{tabular}
\end{center}
\nln

Como no sabemos el estado inicial de \mono{a}, usamos la metavariable $A_0$ para indicar su valor inicial.
Al proceso de ir evaluando consecutivamente los estados en función de las asignaciones que realiza un programa, lo llamamos \textit{transformación} de estados.
A través de la transformación de estados podemos determinar si un programa es correcto o no, dependiendo de su especificación.

\nln

Concretamente, decimos que un programa es \textbf{correcto} respecto de su especificación, si el programa comienza en un estado que cumpla la precondición, y termine
su ejecución en un estado final que satisface la postcondición. Si el programa es correcto, entonces lo expresamos con la tripla de Hoare:

\begin{equation}
    \{P\}S\{Q\}
\end{equation} \label{hoare_triplet}

Donde $(P,Q)$ es la especificación, $P$ es la precondición, $Q$ es la postcondición y $S$ es el programa correcto.

Para ejemplificar, tomemos la siguiente especificación:

\begin{proc}{incrementar}{\Inout a: \Z}{}
    \pre{a = A_0}
    \post{a = A_0 + 1}
\end{proc}

\nln

Y para ésta especificación, implementamos el siguiente programa:


\begin{proc}{incrementar}{\Inout a: \Z}{}
    \mono{c := a+2;}       \par
    \mono{result := c-1;}  \par
    \mono{a := result}     \par
\end{proc} \label{specification_example_1}


Pero por lo que hicimos \hyperref[correct_program_example]{previamente}, vemos que el resultado de este programa es:

\begin{center}
{\textcolor{blue}{\mono{\textbraceleft a = \ensuremath{A_0} \y\;c = \ensuremath{A_0} + 2 \y\;result = (\ensuremath{A_0} + 2) - 1 = \ensuremath{A_0} + 1\textbraceright}}}
\end{center}

\nln

Por lo que el programa es correcto, ya que cumple con todo lo pedido. En particular, el resultado cumple la postcondición.

Ahora bien, teniendo un programa correcto respecto de su especificación, nos gustaría decir lo podemos escribir en forma de tripla de Hoare $\{P\}S\{Q\}$, es decir, queremos
escribir algo del estilo:

\begin{proc}{incrementar}{\Inout a: \Z}{}
    \pre{a = A_0}
    \mono{c := a+2;}       \par
    \mono{result := c-1;}  \par
    \mono{a := result}     \par
    \post{a = A_0 + 1}
\end{proc} \label{specification_example_1_triplet}

Sin embargo, para poder hacerlo, debemos corroborar que sea una tripla \ul{válida}. Decimos que una tripla de Hoare es válida si la precondición es la menos restrictiva para que se
asegure la postcondición al final del programa. Ésta condición de \textit{menos restrictiva} tiene una notación especial, decimos que es la \textit{precondición más debil}, y la denotamos:

\begin{align}
    & \text{Predicado $P$ más debil para que $S$ cumpla $Q$} \Iff \text{wp}(S, Q) \label{wp_definition} \\
    & \text{La tripla $\{P\}S\{Q\}$ es válida} \Iff P \implicaLuego \text{wp}(S,Q) \label{wp_validity}
\end{align}

Ahora bien, como sabemos que el programa implementado para la \hyperref[specification_example_1]{especificación} es correcto, es decir que se cumple la postcondición si se cumple la precondición,
y además la precondición no restringe la entrada, podemos decir que la \hyperref[specification_example_1_triplet]{tripla} es válida. Pero, ¿Cómo haríamos con \textit{cualquier} especificación y programa
correspondiente?

Tomemos ahora como ejemplo la siguiente tripla:

\begin{proc}{mayorIgualaSiete}{\Inout x: \Z}{}
    \pre{P}
    \mono{y := 2*x;}    \par
    \mono{x := y+1}     \par
    \post{Q: x \geq 7}
\end{proc} \label{specification_example_2_triplet}

¿Qué precondición $P$ hace que la tripla sea válida? Es decir, ¿Cuál es la precondición más débil de $S$ para que se cumpla $Q$?

\begin{equation*}
    \text{$P$ = \text{wp}($S$,$Q$)}
\end{equation*}

Para poder deducir esto, necesitamos usar los axiomas de la precondición más debil:

\begin{enumerate}
    \item wp(\mono{x := E}, $Q$) $\equiv$ def($E$) $\yLuego$ $Q_{E}^x$
    \item wp(\mono{skip}, $Q$) $\equiv$ $Q$
    \item wp($S1$\mono{;}$S2$, $Q$) $\equiv$ wp($S1$, wp($S2$,$Q$))
\end{enumerate}

\newpage Acá introdujimos un poco de notación nueva:

\begin{itemize}
    \item def($E$) -- Son las condiciones necesarias para que $E$ esté definida. Por ejemplo,
    \begin{itemize}[label=$\cdot$]
        \item def($x+y$) = def($x$) $\land$ def($y$)
        \item def($\frac{x}{y}$) = def($x$) $\land$ (def($y$) $\yLuego$ $y \neq 0$)
        \item def($\sqrt{x}$) = def($x$) $\yLuego$ $x \geq 0$
        \item def($a[i]+3$) = (def($a$) $\land$ def($i$) $\yLuego$ $0 \leq i < |a|$)
    \end{itemize}
    Si además asumimos que def($a$) = True para cualquier $a$ variable, entonces:
    \begin{itemize}[label=$\cdot$]
        \item def($x+y$) = True
        \item def($\frac{x}{y}$) = $y \neq 0$
        \item def($\sqrt{x}$) = $x \geq 0$
        \item def($a[i]+3$) = $0 \leq i < |a|$
    \end{itemize} \nln

    \item $Q_{E}^x$ -- Es el predicado que se obtiene reemplazando en $Q$ todas las apariciones \textbf{libres} de $x$ por $E$. Por ejemplo,  
    \newcounter{saveeqn} % Create new equation counter
    \setcounter{saveeqn}{\value{equation}} % Set the counter
    \setcounter{equation}{0} % Reset the counter
    \renewcommand{\theequation}{\Alph{equation}} % Change the equation numbering macro
    \begin{align}
        Q &\equiv 0 \leq i < j < n \yLuego a[i] \leq x < a[j]                   \nonumber \\
        \rightarrow\;\; Q_{i+1}^i &\equiv 0 \leq i+1 < j < n \yLuego a[i+1] \leq x < a[j] \\
        \nonumber \\
        \nonumber \\
        Q &\equiv 0 \leq i < n \yLuego (\forall j : \Z)(a[j] = x)               \nonumber \\
        \rightarrow\;\; Q_{k}^{j} &\equiv 0 \leq i < n \yLuego (\forall j : \Z)(a[j] = x) %%
        \label{qex_bound_variable}
    \end{align}
    \setcounter{equation}{\value{saveeqn}}
    \renewcommand{\theequation}{\alph{equation}} % Reset the equation numbering
\end{itemize}

Nótese que en la relación de equivalencia \ref{qex_bound_variable}, no podemos cambiar la variable $j$ como lo indicaría el predicado, pues $j$ es es una variable ligada y no libre.
\nln
Teniendo esto en cuenta, estamos listos para determinar en la \hyperref[specification_example_2_triplet]{tripla anterior} cuál es la precondicion $P$ más débil para que la tripla sea válida.
Empecemos analizando el programa desde el fin de la ejecución, es decir, desde la postcondición, hacia el principio:

\begin{center}
    \textcolor{blue}{\{$Q: x \geq 7$\}} \\
    \mono{x := y+1} \\
    \textcolor{blue}{\{wp(\mono{x:=y+1}, $Q$)\}}
\end{center}
\nln

Pero, por lo que vimos previamente sabemos que:

\begin{equation*}
    \T{wp(\mono{x:=y+1}, $Q$)} \equiv \T{def($y+1$) $\yLuego$ $y+1 \geq 7$} \equiv y \geq 6
\end{equation*}

Por ende, 

\begin{center}
    \textcolor{blue}{\{$Q: x \geq 7$\}}   \\
    \mono{x := y+1}                       \\
    \textcolor{blue}{\{$y \geq 6$\}}      \\
    \mono{y := 2*x;}                      \\
    \textcolor{blue}{\{$P$\}}             \\
\end{center}
\nln

Y vemos que el valor de $P$ debe ser, según lo visto:

\begin{equation*}
    P = \T{wp(\mono{x:=2*x}, $R: y \geq 6$) $\equiv$ def(\mono{2*x}) $\yLuego$ $R_{\T{\mono{2*x}}}^y$ $\equiv$ $x \geq 3$} 
\end{equation*}

Así, ya podemos reescribir las condiciones como una transformación de estados en orden, queda tal que:

\begin{center}
    \textcolor{blue}{\{$P: x \geq 3$\}}   \\
    \mono{y := 2*x;}                      \\
    \textcolor{blue}{\{$R: y \geq 6$\}}   \\
    \mono{x := y+1}                       \\
    \textcolor{blue}{\{$Q: x \geq 7$\}}   \\
\end{center}
\nln

Y además, la tripla de Hoare válida para \mono{mayorIgualaSiete} queda:

\begin{proc}{mayorIgualaSiete}{\Inout x: \Z}{}
    \pre{P: x \geq 3}
    \mono{y := 2*x;}    \par
    \mono{x := y+1}     \par
    \post{Q: x \geq 7}
\end{proc} \label{specification_example_2_final}

\subsection{Condicionales -- Alternativas}

Lo visto hasta ahora corresponde a la corrección de programas cuyo flujo de control es sólo secuencial, veamos ahora el caso de los condicionales.
Tomemos como ejemplo la siguiente tripla:

\begin{proc}{alternativa}{\In a: \Z, \Out c: \Z}{}
    \pre{P: a = A_0}
    \mono{if} a\;$>$\;0  \par
    \;\;\mono{c := a;}   \par
    \mono{else}          \par
    \;\;\mono{c := -a}   \par
    \mono{endif}         \par
    \post{Q: c = |a|}
\end{proc} \label{specification_example_3_triplet}

¿Es correcta la postcondición? ¿Cómo podemos saberlo? Debemos corroborar que se cumpla en \ul{ambos casos}.

\begin{itemize}
    \item \textbf{Rama positiva -- $a > 0$}
          \begin{itemize}[label=$\;$]
            \item Se cumple la condición $\rightarrow$ \textcolor{blue}{\{$a = A_0$ $\land$ $a > 0$\} $\equiv$ \{$a = A_0$ $\land$ $A_0 > 0$\}}
            \item \mono{c := a;} $\rightarrow$ \textcolor{blue}{\{$a = A_0$ $\land$ $c = A_0$ $\land$ $A_0 > 0$\}}
            \item[$\Rightarrow$] \textcolor{blue}{\{$c = |a|$\}}
          \end{itemize}
    \item \textbf{Rama negativa -- $a \leq 0$}
    \begin{itemize}[label=$\;$]
        \item No se cumple la condición $\rightarrow$ \textcolor{blue}{\{$a = A_0$ $\land$ $\neg(a > 0)$\} $\equiv$ \{$a = A_0$ $\land$ $A_0 \leq 0$\}}
        \item \mono{c := a;} $\rightarrow$ \textcolor{blue}{\{$a = A_0$ $\land$ $c = -A_0$ $\land$ $A_0 \leq 0$\}}
        \item[$\Rightarrow$] \textcolor{blue}{\{$c = |a|$\}}
      \end{itemize}
\end{itemize}

Como en ambos casos vale que \textcolor{blue}{\{$c = |a|$\}} entonces la condición también vale luego del condicional y por ende la postcondición se cumple.
\nln 
Pero ahora, ¿Cómo podemos saber si la tripla es válida? Es decir, teniendo un condicional, ¿Cómo podemos saber cuál es su precondición más debil? Para ésto
necesitamos añadir otro axioma de la precondición más debil.

\begin{align}
    &\text{Si \mono{S = if B then S1 else S2 endif}, entonces} \nonumber \\
    &\text{wp}(S,Q)\;\equiv\;\text{def}(B)\;\yLuego\;((B\;\land\;\text{wp}(\text{\mono{S1}}, Q))\;\lor\;(\neg B\;\land\;\text{wp}(\text{\mono{S2}}, Q)))
\end{align} \label{conditional_wp_axiom}

Luego, de este axioma, se deduce el siguiente Teorema:

\begin{align}
    & \text{Si } P \rightarrow \text{def}(B) \text{ y las valen las triplas:} \nonumber \\
    & \{P \land B\}\text{\mono{S1}} \{Q\} \nonumber \\
    & \{P \land \neg B\}\text{\mono{S2}} \{Q\} \nonumber \\
    & \nonumber \\
    & \text{Entonces, } \nonumber \\
    & \{P\} \text{\mono{if B then S1 else S2 endif}} \{Q\}
\end{align} \label{conditional_wp_theorem}

\newpage
Analicemos por ejemplo la siguiente tripla: \nln
\begin{proc}{alternativa}{\In x: \Z, \Out y: \Z}{}
    \pre{P}
    \mono{if} x\;$>$\;0  \par
    \;\;\mono{y := x;}   \par
    \mono{else}          \par
    \;\;\mono{y := -x}   \par
    \mono{endif}         \par
    \post{Q: y \geq 2}
\end{proc} \label{alternative_example_triplet}

¿Qué condición $P$ será la más debil del condicional? Y, ¿Cómo podemos probar que la tripla obtenida es correcta?

Priemro, para determinar wp($S$,$Q$) usamos el axioma visto \hyperref[conditional_wp_axiom]{recientemente}:

\begin{align*}
    \T{wp}(S,Q) &\equiv \T{def}(x > 0)\;\yLuego\;(x > 0\;\land\;x \geq 2)\;\lor\;(x \leq 0\;\land\;-x \geq 2) \\
                &\equiv (x \geq 2) \lor (x \leq -2) \\
                &\equiv |x| \geq 2 \rightarrow \text{\textcolor{blue}{\{$P: |x| \geq 2$\}}}
\end{align*}

Esto nos da la siguiente tripla:

Analicemos por ejemplo la siguiente tripla: \nln
\begin{proc}{alternativa}{\In x: \Z, \Out y: \Z}{}
    \pre{P: |x| \geq 2}
    \mono{if} x\;$>$\;0  \par
    \;\;\mono{y := x;}   \par
    \mono{else}          \par
    \;\;\mono{y := -x}   \par
    \mono{endif}         \par
    \post{Q: y \geq 2}
\end{proc} \label{alternative_example_triplet_2}

Corroboremos su validez usando el \hyperref[conditional_wp_theorem]{Teorema anterior}:

% TODO: Cambiar el color verde por otro verde un poco más oscuro, éste no se ve nada lmao
\begin{align*}
                 P \land B &\implicaLuego \T{wp}(\T{\mono{y := x}},Q) \\
    |x| \geq 2 \land x > 0 &\implicaLuego \T{def}(x) \yLuego x \geq 2 \equiv x \geq 2 \\ 
    |x| \geq 2 \land x > 0 &\implicaLuego x \geq 2 \implies \T{\textcolor{teal}{Verdadero \checkmark}}\\ 
                                                                                      \\
           P \land \neg B  &\implicaLuego \T{wp}(\T{\mono{y := -x}},Q) \\
    |x| \geq 2 \land x \leq 0 &\implicaLuego \T{def}(x) \yLuego -x \geq 2 \equiv x \leq 2 \\
    |x| \geq 2 \land x \leq 0 &\implicaLuego x \leq 2  \implies \T{\textcolor{teal}{Verdadero \checkmark}} \\ 
\end{align*}

Por ende, como ambas triplas valen entonces el condicional es correcto, y la tripla dada previamente es válida.

% TODO: Corrección de secuencias

\subsection{Ciclos -- Teorema del Invariante}

Ahora es el turno de los ciclos, ¿Cómo podemos saber si son correctos o no? Recordemos que un ciclo en SmallLang se ve algo así:

\begin{center}
    \begin{tabular}{l}
        \mono{while (B)}           \\ 
        \mono{S;}                  \\ 
        \mono{endwhile}
    \end{tabular}
\end{center}

Donde \mono{B} es la \textit{guarda} y es una expresión lógica. Mientras \mono{B} sea verdadera, se repite el cuerpo del ciclo, que en este caso ejecuta
al programa \mono{S}. A cada ejecución la llamamos \textit{iteración}. Decimos que el ciclo es correcto si \ul{termina} y además se cumple la \ul{invariante del ciclo.}
\nln
En particular, para que un ciclo sea correcto debemos demostrar que:
\begin{enumerate}
    \item Un predicado invariante $I$ es verdadero antes de que comience el ciclo, tal que $P \rightarrow I$.
    \item La tripla $\{I \land B\}S_i\{I\}$ se cumple para cada instrucción $S_i$ del ciclo.
    \item El predicado $P \land \neg B \rightarrow Q$ es verdadero. Es decir, que se cumpla la postcondición deseada.
    \item El predicado $P \land B \rightarrow (t > 0)$ donde $t$ es la cantidad de iteraciones. Es decir, queremos ver que $t$ esté limitada inferiormente por 0, mientras
    el ciclo no haya terminado.
    \item La tripla $\{I \land B\}$\mono{t1 = t;}$S\;$$\{t < t1\}$ valga para $1 \leq i \leq n$ de forma tal que esté garantizado que $t$ decrezca por cada iteración del ciclo.
\end{enumerate} \cite{gries} \label{correct_cycles_theorem}

Las primeras tres demostraciones nos hablan de si se cumple la invariante $I$, y juntas forman el \textbf{Teorema del Invariante}. Si demostramos el Teorema del Invariante para
un ciclo, entonces habremos demostrado que el ciclo es \textit{parcialmente} correcto. Es decir, demostramos que si el ciclo termina, entonces cumplirá con $Q$. No obstante, las últimas
dos demostraciones nos hablan de la seguridad de que el ciclo termine. Juntando todo, si demostramos que un ciclo cumple el Teorema del Invariante, y además demostramos que el ciclo
siempre termina, entonces el ciclo será correcto.

% TODO: wp de ciclos


\end{document}